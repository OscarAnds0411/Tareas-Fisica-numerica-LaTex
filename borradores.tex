\documentclass[12pt]{article}
\usepackage[utf8]{inputenc}
\usepackage{amsmath,amssymb,amsfonts}
\usepackage{graphicx}
\usepackage{siunitx}
\usepackage{hyperref}
\usepackage{caption}
\usepackage{booktabs}
\title{Física Numérica --- Tarea \#2\\Reporte de laboratorio}
\author{Alumno: (tu nombre) \\ Curso: Física Numérica}
\date{\today}

\begin{document}
\maketitle

\section*{Referencia}
Este informe responde a las indicaciones de la Tarea \#2 (enunciado provisto). :contentReference[oaicite:1]{index=1}

\section{Objetivos}
\begin{itemize}
  \item Reescribir expresiones susceptibles a \emph{cancelación sustractiva} para los argumentos indicados.
  \item Implementar y comparar métodos de recursión hacia arriba y hacia abajo para calcular las funciones de Bessel esféricas $j_{\ell}(x)$ para los primeros 25 valores de $\ell$ en $x=0.1,\,1,\,10$.
  \item Ajustar el programa para que al menos un método alcance error relativo $\le 10^{-10}$ y comparar errores relativos entre métodos.
\end{itemize}

\section{1. Cancelación sustractiva}
La \emph{cancelación sustractiva} ocurre cuando restas dos números próximos y pierdes dígitos de precisión. Para evitarla se suelen reescribir las expresiones usando identidades algebraicas o series asintóticas.

\subsection*{(a) \quad $\sqrt{x+1}-1$; \quad $x\approx 0$}
Reescritura para evitar la resta directa:
\[
\sqrt{x+1}-1 = \frac{(\sqrt{x+1}-1)(\sqrt{x+1}+1)}{\sqrt{x+1}+1}
= \frac{x}{\sqrt{x+1}+1}.
\]
Para $x\to 0$ la forma $\dfrac{x}{\sqrt{x+1}+1}$ evita cancelación al tener división por cantidad $\approx 2$.

\subsection*{(b) \quad $\sin x - \sin y$; \quad $x\approx y$}
Usando identidad trigonométrica:
\[
\sin x - \sin y = 2\cos\!\left(\frac{x+y}{2}\right)\sin\!\left(\frac{x-y}{2}\right).
\]
Si $x\approx y$ entonces $\sin\!\left(\frac{x-y}{2}\right)$ es pequeño pero no hay resta direc­ta de dos cantidades casi iguales: se reduce el problema numérico.

\subsection*{(c) \quad $x^2 - y^2$; \quad $x\approx y$}
Factorizando:
\[
x^2-y^2 = (x-y)(x+y).
\]
Multiplicar por $(x+y)$ conocido evita restar cuadrados casi iguales.

\subsection*{(d) \quad $\dfrac{1-\cos x}{\sin x}$; \quad $x\approx 0$}
Usando la identidad $1-\cos x = 2\sin^2(x/2)$:
\[
\frac{1-\cos x}{\sin x} = \frac{2\sin^2(x/2)}{\sin x} = \frac{2\sin(x/2)}{2\cos(x/2)} = \tan\!\left(\frac{x}{2}\right)\cdot \frac{\sin(x/2)}{\cos(x/2)}?
\]
Mejor proceder directamente:
\[
\frac{1-\cos x}{\sin x} = \frac{2\sin^2(x/2)}{2\sin(x/2)\cos(x/2)} = \tan\left(\frac{x}{2}\right).
\]
Así
\[
\frac{1-\cos x}{\sin x}=\tan\!\left(\frac{x}{2}\right),
\]
que para $x\to 0$ es estable (tiene expansión $\approx x/2$).

(Nota: una forma alternativa es dividir numerador y denominador por $x$ y usar series.)

\subsection*{(e) \quad $c = \sqrt{a^2 + b^2 - 2ab\cos\theta}$; \quad $a\approx b$, $|\theta|\ll 1$}
Esta es la ley de cosenos. Cuando $a\approx b$ y $\theta$ pequeño, hay cancelación entre $a^2+b^2$ y $2ab\cos\theta$. Reescribimos usando factorización:
\[
c = \sqrt{(a-b)^2 + 2ab(1-\cos\theta)}.
\]
Y usando $1-\cos\theta = 2\sin^2(\theta/2)$:
\[
c = \sqrt{(a-b)^2 + 4ab\sin^2(\tfrac{\theta}{2})}.
\]
Si $a=b$ esta forma reduce al caso $c = 2a \sin(\tfrac{|\theta|}{2})$, totalmente estable. En el caso $a\approx b$ usar la forma anterior evita restas que anulan dígitos.

\subsection*{Comentarios numéricos}
En el código se muestran comparaciones numéricas entre evaluación directa y evaluación reescrita para ilustrar la ganancia en precisión.

\section{2. Funciones de Bessel esféricas $j_\ell(x)$}
\subsection{Definición y recurrencias}
Las funciones de Bessel esféricas se definen (una forma) por
\[
j_\ell(x)=\sqrt{\frac{\pi}{2x}}J_{\ell+\tfrac12}(x),
\]
pero para cómputo recursivo se usan las recurrencias:
\[
\boxed{ \; j_{\ell+1}(x)=\frac{2\ell+1}{x}\,j_\ell(x)-j_{\ell-1}(x) \quad (\text{hacia arriba}) \;}
\]
\[
\boxed{ \; j_{\ell-1}(x)=\frac{2\ell+1}{x}\,j_\ell(x)-j_{\ell+1}(x) \quad (\text{hacia abajo}) \;}
\]

\subsection{Método hacia arriba (up)}
\begin{enumerate}
  \item Se conocen $j_0(x)=\dfrac{\sin x}{x}$ y $j_1(x)=\dfrac{\sin x}{x^2}-\dfrac{\cos x}{x}$.
  \item Aplicar la fórmula hacia arriba para obtener $j_2,\ldots,j_{L_{\max}}$.
\end{enumerate}
Este método es simple y preciso para órdenes $\ell$ no demasiado grandes comparado con $x$, pero puede volverse numéricamente inestable cuando $\ell$ crece mucho (dependiendo de $x$).

\subsection{Método hacia abajo (down) con normalización}
\begin{enumerate}
  \item Elegir $L \gg L_{\max}$ (p.e. $L_{\text{start}} = L_{\max}+N_{\text{pad}}$ con $N_{\text{pad}}\approx 50$--$200$ dependiendo de $x$).
  \item Poner valores arbitrarios iniciais, por ejemplo $y_{L_{\text{start}}+1}=0$, $y_{L_{\text{start}}}=1$.
  \item Usar la recurrencia hacia abajo
  \[
  y_{\ell-1}=\frac{2\ell+1}{x}y_\ell - y_{\ell+1}
  \]
  hasta $\ell=0$. Los $y_\ell$ obtenidos son proporcionales a los $j_\ell(x)$ reales.
  \item Normalizar toda la secuencia imponiendo
  \[
  \alpha = \frac{j_0(x)_{\text{exact}}}{y_0},
  \quad \text{con } j_0(x)_{\text{exact}}=\frac{\sin x}{x},
  \]
  y entonces $j_\ell(x)=\alpha\, y_\ell$.
\end{enumerate}
Este procedimiento es numéricamente estable y suele producir resultados precisos incluso para órdenes grandes.

\subsection{Ajuste para error relativo $\le 10^{-10}$}
El parámetro a controlar es $N_{\text{pad}}$ (cuánto más grande, mejor la estabilidad de la normalización). En el código se incluye una búsqueda/ajuste automático que incrementa $N_{\text{pad}}$ hasta alcanzar el criterio de error relativo con respecto a una referencia (en nuestro caso la referencia es la evaluación con la normalización y $N_{\text{pad}}$ grande o la función de referencia explícita cuando está disponible).

\subsection{Comparaciones y métricas}
Para cada $x$ y cada $\ell$ imprimimos:
\[
\frac{\left|j_\ell^{(\text{up})}-j_\ell^{(\text{down})}\right|}
{\left|j_\ell^{(\text{up})}\right|+\left|j_\ell^{(\text{down})}\right|}\quad\text{(medida simétrica)},
\]
y el \emph{error relativo} con respecto a la referencia.

\section{Resultados (resumen)}
En el archivo de salida (el programa) se muestran tablas con:
\begin{itemize}
  \item Valores de $j_\ell(x)$ calculados por el método \texttt{up} y por \texttt{down} para $\ell=0,\ldots,24$ y $x=\{0.1,\,1,\,10\}$.
  \item Columnas con las diferencias absolutas y relativas, y la métrica solicitada:
  \[
  \frac{\ell\,|j_\ell^{\text{(up)}}-j_\ell^{\text{(down)}}|}{|j_\ell^{\text{(up)}}|+|j_\ell^{\text{(down)}}|}
  \]
  (seguí la forma pedida en el enunciado).
\end{itemize}

\section{Discusión y crítica}
\begin{itemize}
  \item Las reescrituras algebraicas para evitar cancelación son estándar y reducen significativamente la pérdida de precisión en los ejemplos numéricos provistos.
  \item Para las funciones de Bessel esféricas, la técnica de recurrencia hacia abajo con normalización es la más robusta para obtener alta precisión en un rango amplio de $\ell$ y $x$. La elección del valor de partida $L_{\text{start}}$ (o el ``padding'') es crucial: valores demasiado pequeños degradan la precisión; valores mayores incrementan costo computacional pero garantizan precisión.
  \item El método hacia arriba es simple y eficiente cuando $\ell\lesssim x$ pero puede producir errores acumulativos para $\ell\gg x$.
\end{itemize}

\section{Conclusiones}
Se entregó una solución numérica completa que incluye reescrituras para evitar cancelación y la implementación de métodos up/down para $j_\ell(x)$ con criterios de precisión. El código adjunto permite reproducir todas las tablas y figuras solicitadas.

\section*{Anexo: Instrucciones para ejecutar los códigos}
\begin{enumerate}
  \item Guardar el archivo \texttt{tarea2.py} (adjunto) en el directorio de trabajo.
  \item Ejecutar con \texttt{python3 tarea2.py}. El programa imprimirá tablas y guardará (si se desea) archivos de salida.
  \item El código está comentado y separado en secciones (cancelaciones, bessel up, bessel down, comparación).
\end{enumerate}

\end{document}
