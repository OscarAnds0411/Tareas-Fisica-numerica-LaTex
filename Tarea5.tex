\documentclass[10pt]{extarticle}

\usepackage[spanish]{babel}
\usepackage[T1]{fontenc}
\usepackage{lmodern,mathrsfs}
\usepackage{amssymb}
\usepackage{xparse}
\usepackage[inline,shortlabels]{enumitem}
\setlist{topsep=2pt,itemsep=2pt,parsep=0pt,partopsep=0pt}
\usepackage[dvipsnames]{xcolor}
\usepackage[utf8]{inputenc}
\usepackage[a4paper,top=0.5in,bottom=0.2in,left=0.5in,right=0.5in,footskip=0.3in,includefoot]{geometry}
\usepackage[most]{tcolorbox}
\tcbuselibrary{minted} % tcolorbox minted library, required to use the "minted" tcb listing engine (this library is not loaded by the option [most])
\usepackage{minted} % Allows input of raw code, such as Python code
\usepackage{newunicodechar}
\newunicodechar{₀}{$_0$}
\usepackage{minted} % Allows input of raw code, such as Python code
\usepackage[hidelinks]{hyperref} % ALWAYS load this package LAST
\usepackage{listings}
\usepackage{diffcoeff} % for easy edo type
\lstset{
    basicstyle=\ttfamily\footnotesize,
    keywordstyle=\color{blue},
    commentstyle=\color{gray},
    stringstyle=\color{green!40!black},
    showstringspaces=false,
    numbers=left,
    numberstyle=\tiny,
    frame=single,
    breaklines=true
}

% Custom tcolorbox style for Python code (not the code or the box it appears in, just the options for the box)
\tcbset{
    pythoncodebox/.style={
        enhanced jigsaw,breakable,
        colback=gray!10,colframe=gray!20!black,
        boxrule=1pt,top=2pt,bottom=2pt,left=2pt,right=2pt,
        sharp corners,before skip=10pt,after skip=10pt,
        attach boxed title to top left,
        boxed title style={empty,
            top=0pt,bottom=0pt,left=2pt,right=2pt,
            interior code={\fill[fill=tcbcolframe] (frame.south west)
                --([yshift=-4pt]frame.north west)
                to[out=90,in=180] ([xshift=4pt]frame.north west)
                --([xshift=-8pt]frame.north east)
                to[out=0,in=180] ([xshift=16pt]frame.south east)
                --cycle;
            }
        },
        title={#1}, % Argument of pythoncodebox specifies the title
        fonttitle=\sffamily\bfseries
    },
    pythoncodebox/.default={}, % Default is No title
    %%% Starred version has no frame %%%
    pythoncodebox*/.style={
        enhanced jigsaw,breakable,
        colback=gray!10,coltitle=gray!20!black,colbacktitle=tcbcolback,
        frame hidden,
        top=2pt,bottom=2pt,left=2pt,right=2pt,
        sharp corners,before skip=10pt,after skip=10pt,
        attach boxed title to top text left={yshift=-1mm},
        boxed title style={empty,
            top=0pt,bottom=0pt,left=2pt,right=2pt,
            interior code={\fill[fill=tcbcolback] (interior.south west)
                --([yshift=-4pt]interior.north west)
                to[out=90,in=180] ([xshift=4pt]interior.north west)
                --([xshift=-8pt]interior.north east)
                to[out=0,in=180] ([xshift=16pt]interior.south east)
                --cycle;
            }
        },
        title={#1}, % Argument of pythoncodebox specifies the title
        fonttitle=\sffamily\bfseries
    },
    pythoncodebox*/.default={}, % Default is No title
}

% Custom tcolorbox for Python code (not the code itself, just the box it appears in)
\newtcolorbox{pythonbox}[1][]{pythoncodebox=#1}
\newtcolorbox{pythonbox*}[1][]{pythoncodebox*=#1} % Starred version has no frame

% Custom minted environment for Python code, NOT using tcolorbox
\newminted{python}{autogobble,breaklines,mathescape}

% Custom tcblisting environment for Python code, using the "minted" tcb listing engine
% Adapted from https://tex.stackexchange.com/a/402096
\NewTCBListing{python}{ !O{} !D(){} !G{} }{
    listing engine=minted,
    listing only,
    pythoncodebox={#1}, % First argument specifies the title (if any)
    minted language=python,
    minted options/.expanded={
        autogobble,breaklines,mathescape,
        #2 % Second argument, delimited by (), denotes options for the minted environment
    },
    #3 % Third argument, delimited by {}, denotes options for the tcolorbox
}

%%% Starred version has no frame %%%
\NewTCBListing{python*}{ !O{} !D(){} !G{} }{
    listing engine=minted,
    listing only,
    pythoncodebox*={#1}, % First argument specifies the title (if any)
    minted language=python,
    minted options/.expanded={
        autogobble,breaklines,mathescape,
        #2 % Second argument, delimited by (), denotes options for the minted environment
    },
    #3 % Third argument, delimited by {}, denotes options for the tcolorbox
}

% verbbox environment, for showing verbatim text next to code output (for package documentation and user learning purposes)
\NewTCBListing{verbbox}{ !O{} }{
    listing engine=minted,
    minted language=latex,
    boxrule=1pt,sidebyside,skin=bicolor,
    colback=gray!10,colbacklower=white,valign=center,
    top=2pt,bottom=2pt,left=2pt,right=2pt,
    #1
} % Last argument allows more tcolorbox options to be added

\setlength{\parindent}{0.2in}
\setlength{\parskip}{0pt}
\setlength{\columnseprule}{0pt}

\makeatletter
% Redefining the title block
\renewcommand\maketitle{
    \null\vspace{4mm}
    \begin{center}
        {\Huge\sffamily\bfseries\selectfont\@title}\\
            \vspace{4mm}
        {\Large\sffamily\selectfont\@author}\\
            \vspace{4mm}
        {\large\sffamily\selectfont\@date}
    \end{center}
    \vspace{6mm}
}
% Adapted from https://tex.stackexchange.com/questions/483953/how-to-add-new-macros-like-author-without-editing-latex-ltx?noredirect=1&lq=1
\makeatother

\title{Tarea \#5 Física numérica}
\author{Oscar Andrés Valencia Magaña}
\date{\today}
% Created April 6, 2023

\begin{document}
\maketitle
\tableofcontents
\listoffigures
\section{Introducción}
A lo largo del semestre se estudiaron de manera sistemática diversas técnicas de ajuste de datos, así como los criterios necesarios para evaluar la calidad y la validez de dichos ajustes en el análisis de resultados experimentales. Estas herramientas permiten identificar el modelo matemático que mejor describe un fenómeno físico, cuantificar la incertidumbre asociada y determinar la confiabilidad de las conclusiones obtenidas a partir de un conjunto de mediciones.

El propósito de esta tarea es aplicar dichas técnicas a tres conjuntos de datos experimentales previamente proporcionados. Para cada conjunto se seleccionará un modelo apropiado, se implementará el proceso de ajuste y se evaluará su bondad mediante los métodos revisados en clase.

\section{Interpolación de datos y ajustes para la resonancia de Breit-Wigner}
Para este punto nos dan los siguientes datos experimentales dados en la siguiente tabla:

\begin{table}[H]
    \centering
    \begin{tabular}{c|ccccccccc}
        \hline
        $i =$ & 1& 2& 3& 4& 5& 6& 7& 8& 9\\
        \hline
        $E_i\, (\text{MeV})$ &0 &25& 50& 75& 100& 125& 150& 175& 200\\
        $f\left(E_i\right) (\text{MeV})$& 10.6 & 16.0 &45.0 &83.5 &52.8 &19.9& 10.8& 8.25& 4.7\\
        $\sigma_i (\text{MeV})$& 9.34 &17.9 &41.5& 85.5& 51.5& 21.5 &10.8& 6.29 &4.14\\
        \hline
    \end{tabular}
    \caption{Datos experimentales para la resonancia de Breit-Wigner.}
    \label{tab:Breit-Wigner-data}
\end{table}
Lo que buscaremos hacer será lo siguiente:
\begin{enumerate}[a)]
    \item Escribir una función en Python que realice un ajuste a un polinomio segun Lagrange y su algoritmo, para un conjunto de $n$ puntos.
    \item Utilizar la función creada anteriormente para ajustar a un polinomio los datos experimentale dados en la tabla \ref{tab:Breit-Wigner-data} y graficarlo en pasos de $5 \text{MeV}$. 
    \item Escribir una función en Python que realice una aproximación para estimar la energia de resonancia $E_r$ y el ancho de la mitad del máximo (\textit{full-width at half-maximum}) \(\Gamma\) y los compararemos con el valor teorico dado \(\left(E_r,\Gamma\right)=(78\text{ MeV}, 55\text{ MeV})\).
    \item Escribir una función en Python que realice un ajuste por Splines cubicos para una conjuntos de $n$ datos dados, posteriormente realizaremos los mismos calculos mencionandos en el punto anterior.
    \item Partiendo de la teoría la resonancia de Breit-Wigner, los datos se deben de ajustar al siguiente modelo:
    \begin{equation}
        f(E) = \frac{f_r}{(E - E_r)^2 + (\Gamma/2)^2}\nonumber
    \end{equation}
    donde $f_r$, $E_r$ y $\Gamma$ son parámetros a determinar mediante el ajuste, minimizando la función $\chi^2$.
    \item Al minimizar $\chi^2$ se obtienen ecuaciones que no son lineales, por lo que escribimos un programa que utilice el método de Newton-Raphson multidimensional para encontrar sus raíces.
\end{enumerate}
\section{Ajuste para un circuito RLC}
\section{¿Es realmente un cuerpo negro?}


\end{document}