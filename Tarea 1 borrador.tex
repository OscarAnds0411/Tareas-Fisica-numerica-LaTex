\documentclass[12pt,a4paper]{article}
\usepackage[utf8]{inputenc}
\usepackage[T1]{fontenc}
\usepackage[spanish]{babel}
\usepackage{amsmath,amssymb,amsthm,physics}
\usepackage{graphicx}
\usepackage{hyperref}
\usepackage{caption}
\usepackage{float}
\usepackage{listings}
\usepackage{xcolor}
\usepackage{siunitx}
\usepackage{geometry}
\geometry{left=2.5cm,right=2.5cm,top=2.5cm,bottom=2.5cm}

\hypersetup{colorlinks=true,linkcolor=blue,urlcolor=blue}

% listings para Python (estilo)
\lstdefinestyle{py}{
  language=Python,
  basicstyle=\small\ttfamily,
  keywordstyle=\color{blue}\bfseries,
  commentstyle=\color{gray}\itshape,
  stringstyle=\color{purple},
  showstringspaces=false,
  breaklines=true,
  frame=single,
  numbers=left,
  numberstyle=\tiny,
  tabsize=4
}

\title{\textbf{Tarea 3 -- Física Numérica}\\
\vspace{6pt}
\large Ecuación de calor 1D y Ecuación de Poisson 2D -- Resolución analítica y numérica}
\author{Oscar Andrés Valencia Magaña\\[2pt]
\small 8FM2 - HECTOR JAVIER URIARTE RIVERA}
\date{7 de Octubre de 2025}

\begin{document}
\maketitle
\begin{center}
\textit{Escuela Superior de Física y Matemáticas, IPN -- Unidad Profesional Zacatenco}
\end{center}
\vspace{12pt}

\begin{abstract}
En este documento se presenta la resolución detallada, tanto analítica como numérica, de dos problemas clásicos de ecuaciones en derivadas parciales:
la ecuación del calor unidimensional con condiciones de Dirichlet homogéneas y la ecuación de Poisson en dos dimensiones con condiciones periódicas. 
Se incluyen derivaciones paso a paso, criterios de estabilidad para los esquemas numéricos, algoritmos implementables en código (Python) y recomendaciones para graficar y analizar resultados.
\end{abstract}

\tableofcontents
\newpage

\section{Introducción}
Este trabajo aborda dos problemas fundamentales en física matemática y computacional: la difusión térmica en una barra y la solución numérica de Poisson en un dominio bidimensional. El objetivo es:
\begin{itemize}
  \item Mostrar la resolución analítica posible.
  \item Diseñar e implementar esquemas numéricos (con detalle paso a paso).
  \item Comparar resultados numéricos con soluciones analíticas cuando existan.
  \item Documentar y entregar el código como apéndice para reproducibilidad.
\end{itemize}

\section{Ecuación de calor en 1D}
\subsection{Planteamiento físico y ecuación gobernante}
Consideramos una barra homogénea de longitud \(L\) aislada lateralmente y con extremos mantenidos a temperatura cero. Sea \(T(x,t)\) la temperatura en la posición \(x\in[0,L]\) y tiempo \(t\ge0\). La ecuación de difusión térmica (ecuación del calor) en 1D es:
\[
\frac{\partial T}{\partial t} = \alpha \frac{\partial^2 T}{\partial x^2},
\]
donde \(\alpha\) es la difusividad térmica del material (\(\alpha = k/(\rho c_p)\)). Impongamos condiciones:
\[
T(x,0)=T_0\quad(\text{condición inicial, constante}),\qquad T(0,t)=0,\quad T(L,t)=0\quad\forall t\ge0.
\]

\subsection{Solución analítica por separación de variables}
Buscamos soluciones de la forma \(T(x,t)=X(x)G(t)\). Insertando en la PDE y separando:
\[
\frac{1}{\alpha}\frac{G'(t)}{G(t)} = \frac{X''(x)}{X(x)} = -\lambda,
\]
donde \(\lambda\) es constante de separación. Las condiciones de frontera \(X(0)=X(L)=0\) conducen a valores propios \(\lambda_n = (n\pi/L)^2,\ n\in\mathbb{N}\) y funciones propias \(X_n(x)=\sin(n\pi x/L)\). Las soluciones temporales son \(G_n(t)=\exp(-\alpha (n\pi/L)^2 t)\). Por superposición, imponiendo la condición inicial constante \(T_0\):
\[
T(x,t)=\sum_{n=1}^{\infty} b_n \sin\!\Big(\frac{n\pi x}{L}\Big) e^{-\alpha (n\pi/L)^2 t},
\]
con coeficientes
\[
b_n = \frac{2}{L}\int_0^L T_0 \sin\!\Big(\frac{n\pi x}{L}\Big)dx
= \frac{4T_0}{n\pi}\quad\text{para }n\text{ impar},\quad b_n=0\ \text{si }n\text{ par}.
\]
Así la solución se expresa (sumando sobre \(n=2k+1\)):
\[
\boxed{T(x,t)=\sum_{k=0}^{\infty}\frac{4T_0}{(2k+1)\pi}\sin\!\Big(\frac{(2k+1)\pi x}{L}\Big)e^{-\alpha\big(\frac{(2k+1)\pi}{L}\big)^2 t}.}
\]

\subsection{Discretización y esquema numérico FTCS (explícito)}
Para resolver numéricamente usamos diferencias finitas en una malla uniforme. Sea \(x_i=i\Delta x\) con \(i=0,\ldots,N\) y \(t^n=n\Delta t\). Aproximamos:
\[
\frac{\partial T}{\partial t}\Big|_{(x_i,t^n)} \approx \frac{T_i^{n+1}-T_i^n}{\Delta t},\qquad
\frac{\partial^2 T}{\partial x^2}\Big|_{(x_i,t^n)} \approx \frac{T_{i+1}^n - 2T_i^n + T_{i-1}^n}{(\Delta x)^2}.
\]
Sustituyendo:
\[
\frac{T_i^{n+1}-T_i^n}{\Delta t} = \alpha \frac{T_{i+1}^n - 2T_i^n + T_{i-1}^n}{(\Delta x)^2}.
\]
Definiendo \(r=\dfrac{\alpha\Delta t}{(\Delta x)^2}\), se obtiene el esquema explícito:
\[
\boxed{T_i^{n+1} = T_i^n + r\big(T_{i+1}^n - 2T_i^n + T_{i-1}^n\big).}
\]

\subsubsection{Condición de estabilidad (análisis de von Neumann)}
Aplicando el análisis de Fourier (von Neumann) al esquema se obtiene la condición de estabilidad:
\[
r \le \frac{1}{2}\quad\Longrightarrow\quad \Delta t \le \frac{(\Delta x)^2}{2\alpha}.
\]
Si \(r>1/2\) la solución numérica se vuelve inestable (creciente / oscilatoria).

\subsection{Algoritmo paso a paso (implementación)}
\begin{enumerate}
  \item Fijar \(L, T_0, \alpha, N\). Calcular \(\Delta x=L/N\).
  \item Elegir \(\Delta t\) tal que \(r\le 1/2\).
  \item Inicializar \(T_i^0=T_0\) para \(i=1,\dots,N-1;\ T_0^n=T_N^n=0\).
  \item Para cada paso de tiempo \(n\): actualizar \(T_i^{n+1}\) mediante la fórmula FTCS para \(i=1,\dots,N-1\).
  \item Calcular error en cada tiempo con respecto a la solución analítica truncada:
  \[
  E_2(t^n)=\sqrt{\frac{1}{N-1}\sum_{i=1}^{N-1}\big(T_i^n - T_{\text{an}}(x_i,t^n)\big)^2 }.
  \]
\end{enumerate}

\subsection{Experimento numérico y observaciones}
\begin{itemize}
  \item \emph{Convergencia:} al refinar \(\Delta x\) y respetar la condición de estabilidad, la solución numérica converge a la solución analítica.
  \item \emph{Violación de estabilidad:} si se elige \(\Delta t\) demasiado grande ( \(r>1/2\) ) se observan oscilaciones crecientes y la solución diverge.
  \item \emph{Equilibrio:} con condiciones de frontera a 0, el equilibrio es \(T(x,t\to\infty)=0\).
\end{itemize}

\subsection{Efecto del material (difusividad térmica)}
Valores típicos (referencia práctica):
\begin{itemize}
  \item Aluminio: \(\alpha_{\mathrm{Al}}\sim 9\times 10^{-5}\ \mathrm{m}^2/\mathrm{s}\).
  \item Madera (mala conductora): \(\alpha_{\mathrm{madera}}\sim 10^{-6}\ \mathrm{m}^2/\mathrm{s}\) (varía).
\end{itemize}
Un menor \(\alpha\) reduce la velocidad de difusión: la barra se enfría mucho más lento. Además la condición de estabilidad limita aún más \(\Delta t\) para una malla dada cuando \(\alpha\) es grande.

\subsection{Recomendaciones numéricas}
\begin{itemize}
  \item Para obtener mayor estabilidad y precisión temporal sin reducir excesivamente \(\Delta t\), usar esquemas implícitos como \emph{BTCS} o \emph{Crank--Nicolson}.
  \item Para comparar con la solución analítica, truncar la serie en un número \(M\) de términos suficiente (por ejemplo \(M=200\)–\(500\)) dependiendo del tiempo considerado.
  \item Visualizar: superficie \(T(x,t)\) 3D, isotermas (contornos) y cortes en tiempos fijos.
\end{itemize}

\section{Ecuación de Poisson en 2D con condiciones periódicas}
\subsection{Planteamiento}
Resolvemos
\[
\frac{\partial^2 \phi}{\partial x^2} + \frac{\partial^2 \phi}{\partial y^2} = f(x,y),
\]
en el dominio \([0,2\pi]\times[0,2\pi]\) con condiciones periódicas en \(x\) y \(y\). Tomamos:
\[
f(x,y)=\cos(3x+4y)-\cos(5x-2y).
\]

\subsection{Discretización y esquema de diferencias finitas}
Tomando una malla uniforme \(N\times N\) con \(\Delta x = \Delta y = 2\pi/N\), la aproximación por diferencias centrales da:
\[
\frac{\phi_{i+1,j}-2\phi_{i,j}+\phi_{i-1,j}}{\Delta x^2} + \frac{\phi_{i,j+1}-2\phi_{i,j}+\phi_{i,j-1}}{\Delta y^2} = f_{i,j}.
\]
Con \(\Delta x=\Delta y\) y reordenando:
\[
\boxed{\phi_{i,j} = \frac{1}{4}\big(\phi_{i+1,j}+\phi_{i-1,j}+\phi_{i,j+1}+\phi_{i,j-1} - (\Delta x)^2 f_{i,j}\big).}
\]
Debido a las condiciones periódicas, los índices se consideran módulo \(N\).

\subsection{Método iterativo: Gauss--Seidel (y SOR opcional)}
\begin{itemize}
  \item \textbf{Gauss--Seidel:} actualiza cada \(\phi_{i,j}\) usando los valores más recientes (in place).
  \item \textbf{SOR:} sobre-relajación sucesiva puede acelerar la convergencia: al actualizar se aplica
  \[
  \phi_{i,j}^{\text{new}} = (1-\omega)\phi_{i,j}^{\text{old}} + \omega \phi_{i,j}^{\text{GS}},
  \]
  donde \(\phi_{i,j}^{\text{GS}}\) es el valor que daría Gauss--Seidel y \(1<\omega<2\) es el parámetro.
\end{itemize}

\subsection{Criterio de convergencia}
Usar la norma del residuo máximo:
\[
\text{res} = \max_{i,j}\Big|\frac{\phi_{i+1,j}-2\phi_{i,j}+\phi_{i-1,j}}{\Delta x^2} + \frac{\phi_{i,j+1}-2\phi_{i,j}+\phi_{i,j-1}}{\Delta y^2} - f_{i,j}\Big|.
\]
Iterar hasta \(\text{res}<\text{tol}\), por ejemplo \(\text{tol}=10^{-8}\).

\subsection{Observaciones y métodos alternativos}
\begin{itemize}
  \item Para condiciones periódicas, la transformada de Fourier discreta (FFT) resuelve el problema de Poisson de manera directa y eficiente ( \(O(N^2\log N)\) ) y evita iteraciones.
  \item Gauss--Seidel es sencillo y robusto; SOR mejora la velocidad si se elige \(\omega\) adecuado.
\end{itemize}

\section{Resultados esperados y forma de presentar}
Para ambos problemas se recomienda presentar:
\begin{itemize}
  \item Gráficas de superficie 3D y de contorno.
  \item Comparación numérico vs analítico (errores \(L^2\) y \(L^\infty\)).
  \item Experimento mostrando inestabilidad cuando se viola la condición de estabilidad (para la ecuación del calor).
  \item Mapas de convergencia (residuo vs iteraciones) para el método de Gauss--Seidel/SOR.
\end{itemize}

\newpage
\appendix
\section{Apéndice A: Código Python -- Ecuación del calor (FTCS) y comparación analítica}
% El siguiente archivo debe guardarse como \texttt{heat_1d_explicit_vs_analytical.py} en la misma carpeta del \texttt{.tex} si se desea incluir mediante \verb|\lstinputlisting|.

% \lstinputlisting[style=py]{heat_1d_explicit_vs_analytical.py}

\section{Apéndice B: Código Python -- Poisson 2D (Gauss--Seidel, periodic)}
% El siguiente archivo debe guardarse como \texttt{poisson_2d_gauss_seidel_periodic.py}.

% \lstinputlisting[style=py]{poisson_2d_gauss_seidel_periodic.py}

\section{Apéndice C: Notas de uso y reproducibilidad}
\begin{itemize}
  \item Requisitos: Python 3.8+, \texttt{numpy}, \texttt{matplotlib}. Opcional: \texttt{mpl\_toolkits} para gráficos 3D.
  \item Para compilar el \texttt{.tex} y que incluya los listados: compilar en la misma carpeta donde estén los \texttt{.py}.
  \item Para problemas grandes aumente N con cuidado (memoria y tiempo). Para Poisson usar FFT si se requiere rapidez.
\end{itemize}

\end{document}
